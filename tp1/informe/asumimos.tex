Para realizar este trabajo, asumimos:
\begin{itemize}
 \item La gestión del padrón de ciudadanos es correcta, en el sentido que se respetan las leyes de creación
 del padrón según corresponda (en cuanto a distancia al lugar de votación, distribución por apellido, etc.).
 Dado que estos datos son ingresados por un agente externo, el sistema se limita a confiar en esta información, 
 y de hecho se basa en ella para realizar otras validaciones.
 
 \item Cualquier  abarca o bien todas las provincias, o bien un subconjunto de provincias, o bien un subconjunto
 de municipios de una misma provincia. No tuvimos en cuenta mayor granularidad.
 
 \item El alcance de la historia que almacena nuestro sistema, mencionada en la sección inicial, fue determinado
 luego de un proceso iterativo de elicitación, y aceptado por el cliente.
 
 \item Las entidades OpcionElectoral y CandidatoElectoral se consideraron entidades débiles debido a que cada
 votación tendrá sus opciones y sus candidatos, y no tiene sentido hablar de opciones o candidatos cuando no
 existe una elección que los contemple.
 
 \item La entidad MesaElectoral representa la mesa física perteneciente a un centro de votación. A pesar de
 ser parte de este centro, el número de mesa es único en todo el sistema dado que actualmente las mesas de
 cualquier votación tienen un identificador independiente del centro.
\end{itemize}






