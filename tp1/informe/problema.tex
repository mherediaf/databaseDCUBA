El gobierno de una nación se acerca a nosotros con el objetivo de crear un sistema de voto electrónico para futuras elecciones.

\subsection{Alcance}
El alcance del sistema será todo tipo de votación nacional, provincial y municipal,
ya sea para la elección de cargos públicos como la realización de consultas populares a la ciudadanía.

En esta primera etapa, en el caso de las elecciones de cargos públicos, sólo tendremos en cuenta un cargo por elección.

Cada votación abarcará ciertos municipios o provincias, pero no ambos a la vez. Esto significa que si una votación es
a nivel provincial, abarcará las provincias correspondientes, pero no se establerecerá una relación directa con los
municipios que integran cada provincia; si una votación abarca una provincia, automáticamente abarcará todos los municipios.

En cuanto a los tipos de votación, en principio estableceremos unos tipos básicos, permitiendo a futuro nuevos
tipos creados por el administrador del sistema.

\subsection{Padrón electoral}
El sistema aquí introducido incluirá la gestión de un padrón electoral, 
que para cada ciudadano indicará la mesa y el centro de votación donde debe votar. Se proveerá una interfaz de alta
y modificación del padrón, que permitirá ingresar nuevos ciudadanos al mismo, como también determinar una fecha de
defunción cuando corresponda. Los datos de los nuevos ciudadanos deberán ser provistos junto con el centro de votación 
y el número de mesa donde voten.

El padrón NO podrá ser modificado durante una votación. Esto se debe a que mientras haya una votación en curso,
el sistema realizará validaciones en cuanto a que cada persona vote donde el padrón lo indique.

\subsection{Datos históricos de las votaciones}
Se almacenará la historia de todas las votaciones realizadas mediante este sistema, considerando fecha de inicio y fin de las mismas,
así como también el alcance y el tipo. Si se tratare de una elección de un cargo público, se preservarán los datos de los candidatos 
(incluyendo el partido político) y los votos recibidos por cada uno en total y en cada mesa. En el caso de las consultas populares se
mantendrán los datos de las distintas opciones y los votos en total y por cada mesa.

Además por cada mesa almacenaremos quiénes fueron las autoridades: presidente, vicepresidente y fiscales.
De todos ellos guardaremos los datos personales y en el caso de los fiscales, el partido político que representaban.

Por último guardaremos la lista de ciudadanos que votó efectivamente en cada mesa en cada votación. Dado que sabemos la 
fecha en que un ciudadano ingresó al padrón y conocemos también las fechas de inicio de las votaciones, podemos determinar si
efectivamente a un ciudadano le correspondía o no votar y si fue o no a votar cuando le correspondía.

Una observación no menor sobre lo antedicho es que no guardamos a quién/qué votó cada ciudadano, sólo registramos el acto de votación.
Esto permite preservar la confidencialidad del voto. La cantidad de votos por candidato o por opción electoral, se conocen por cada
mesa (no por cada votante).

\subsection{Otros datos tenidos en cuenta}
Por cada mesa almacenamos los datos del técnico asignado a la máquina.

Por cada centro de votación almacenamos los datos de las camionetas asignadas al centro,
que reemplazarán máquinas en caso de que se rompan, así como los datos del conductor asignado a cada una de ellas.
